\chapter{Summary and Outlook}

\section{Summary of Key Findings}

This work presented a modular and methodologically transparent pipeline for multi-view dataset generation based on static and dynamic 3D Gaussian Splatting representations.
Leveraging a purpose-built, hardware-synchronized multi-camera system, temporally aligned image sequences were captured from which both static (3DGS) and dynamic (D3DGS) models were reconstructed.
By integrating promptable segmentation (SAM2) and temporal mask propagation into the reconstruction workflow, all exported object models achieve geometric precision, temporal stability, and direct compatibility with downstream compositing.

A central contribution of the proposed system is the unified treatment of static and dynamic Gaussian models within a single compositional framework.
Pipeline~B enables deterministic scene assembly in calibrated camera rigs, producing spatially and temporally coherent outputs including RGB, depth, and instance-level masks.
By explicitly modeling occlusion, object placement, and temporal interpolation, the pipeline allows full control over rendered datasets while maintaining photometric and geometric realism.
This structure bridges the gap between reconstruction-focused Gaussian pipelines and practical dataset generation tools, supporting reproducible, multi-object scenes with synchronized multimodal annotations.

The framework further supports the seamless integration of externally generated 3D assets, such as those exported from \emph{Trellis}, demonstrating that pre-trained or lightweight 3D models can function as first-class entities within Gaussian-based rendering and composition.
As a result, the system generalizes across input modalities, accommodating both data-driven and generative 3D sources without modifications to the rendering logic.

\section{Limitations}

Despite its flexibility, several limitations should be acknowledged.
First, dynamic reconstructions rely on per-frame segmentation masks that, despite prompt propagation, may accumulate small temporal inconsistencies under strong occlusions or motion blur.
This can result in slight spatial drift of reconstructed Gaussians and propagated keypoints, particularly in regions with sparse Gaussian coverage such as knees or hips.

Second, although the compositional framework supports both static and dynamic elements, all rendered scenes currently assume rigid camera calibration and fixed illumination.
Consequently, phenomena such as global illumination, specular reflections, or cast shadows between independently rendered objects are not explicitly modeled, which can limit realism in certain composite configurations.

Third, pose estimation for dynamic objects is derived from Gaussian trajectories rather than explicit joint-based motion models.
While this approach ensures coherent motion propagation, it lacks semantic interpretability and may diverge from true skeletal motion.

Finally, the current implementation is optimized for controlled laboratory captures using pre-calibrated multi-camera rigs.
Generalizing to in-the-wild or handheld capture setups would require additional mechanisms to handle calibration drift, synchronization errors, and varying illumination conditions.

\section{Future Developments and Research Perspectives}

Several avenues remain open for further exploration.
A natural extension involves integrating articulated motion priors and skeleton-aware Gaussian tracking, which could improve alignment of dynamic keypoints and enable physically interpretable pose propagation.
Embedding such priors directly into the D3DGS training or composition stages may strengthen temporal coherence in regions with complex non-rigid motion, such as human limbs or deformable surfaces.

Another promising direction involves scaling the capture and synthesis process toward larger and more diverse datasets.
Automated scene composition, dynamic lighting control, and procedural environment synthesis could further enhance realism and diversity while maintaining annotation consistency.
In parallel, coupling Gaussian-based scene representations with diffusion-based appearance refinement or NeRF-style volumetric augmentation may bridge the remaining visual gap between synthetic and real-world imagery.

Finally, establishing standardized benchmarks for evaluating compositional and temporal consistency in Gaussian-rendered datasets would be highly beneficial.
Such benchmarks could promote reproducibility, enable quantitative analysis of temporal stability, and help define a common evaluation protocol for future research in Gaussian-based dataset generation.

Overall, this work demonstrates that temporally consistent Gaussian representations, when combined with geometry-aware compositing and transparent annotation pipelines, provide a practical and extensible foundation for the next generation of large-scale, multi-view training datasets.